\documentclass[11pt,authoryear]{article}
\pdfoutput = 1
\usepackage{fullpage,amsthm,amsmath,natbib,algorithm,algorithmic,enumitem,afterpage,amssymb,setspace,graphicx,amsfonts,array,verbatim,commath,mathrsfs,bm}
\usepackage[dvipsnames]{xcolor}
\usepackage{tikz}
\usetikzlibrary{arrows,matrix,positioning, fit, shapes.geometric}
\usepackage[hang,flushmargin]{footmisc}
\usepackage[colorlinks=TRUE,citecolor=Blue,linkcolor=BrickRed,urlcolor=PineGreen]{hyperref}

\usepackage{mathtools}
\usepackage{array}
\newcolumntype{L}[1]{>{\raggedright\arraybackslash}p{#1}}

\numberwithin{equation}{section}
\theoremstyle{plain}
\newtheorem{thm}{Theorem}[section]
\newtheorem{proposition}{Proposition}
\newtheorem{definition}{Definition}
\newtheorem{lemma}{Lemma}
\newtheorem{remark}{Remark}

% widebar
\DeclareFontFamily{U}{mathx}{\hyphenchar\font45}
\DeclareFontShape{U}{mathx}{m}{n}{<-> mathx10}{}
\DeclareSymbolFont{mathx}{U}{mathx}{m}{n}
\DeclareMathAccent{\widebar}{0}{mathx}{"73}

\usepackage{xspace}
\newcommand{\susie}{\textsl{SuSiE}\xspace}

\def\bx{\bm x}
\def\btheta{\bm \theta}
\def\be{\bm e}
\def\bs{\bm s}
\def\U{\mathcal U}
\def\SU{\mathcal SU}
\def\SN{\mathcal SN}
\def\G{\mathcal{G}}
\def\Q{\mathcal{Q}}
\def\L{\mathcal{L}}
\def\C{\mathcal{C}}
\def\iid{\sim^\text{\it iid}}

\def\glhat{\hat{g}_{\l}}
\def\gfhat{\hat{g}_{\f}}
\def\l{\bm l}
\def\f{\bm f}
\def\h{\bm h}
\def\p{p}
\def\ash{{\it ash}}
\def\flash{{\it flash}}
\def\ERsq{\widebar{R^2}}
\def\precij{\prec_{ij}}
\def\E{\text{E}}
\def\PIP{\text{PIP}}

\def\Ef{\text{E}_{\qf}(f_j)}
\def\Efsq{\text{E}_{\qf}(f^2_j)}
\def\Efk{\text{E}_{\qf}f_{kj}}
\def\Efksq{\text{E}_{\qf}(f^2_{kj})}
\def\Efkdash{\text{E}_{\qf}f_{k'j}}

\def\El{\text{E}_{\ql}(l_i)}
\def\Elsq{\text{E}_{\ql}(l^2_i)}
\def\Elk{\text{E}_{\ql}l_{ki}}
\def\Elksq{\text{E}_{\ql}(l^2_{ki})}
\def\Elkdash{\text{E}_{\ql}l_{k'i}}
\def\gfk{g_{\f_k}}
\def\glk{g_{\l_k}}
\def\ghk{g_{\h_k}}

\def\pve{\phi}


\def\ebnm{\text{\it EBNM}}
\def\ser{\text{\it SER}}
\def\prec{\bm \prec}
\def\T{\mathcal T}
\def\iter{t}

\def\Q{\mathcal Q}
\def\G{\mathcal G}
\def\g{\bm g}
\def\q{{\bm q}}
\def\vmu{\bm \mu}
\def\vy{\bm y}
\def\vx{\bm x}
\def\mx{\bm X}
\def\vb{\bm b}
\def\bb{\bar{b}}
\def\vbb{\bar{\bm{b}}}
\def\bbb{\bar{b^2}}
\def\vgamma{\bm \gamma}
\def\vtheta{\bm \theta}
\def\vpi{\bm \pi}
\def\ve{\bm e}
\def\vq{\bm q}
\def\vg{\bm g}
\def\vr{\bm r}
\def\mr{\bm R}
\def\rv{\sigma^2} % residual variance
\def\vrv{\bm \rv}
\def\vmu{\bm \mu}
\def\vsigma{\bm \sigma}
\def\valpha{\bm \alpha}
\def\Var{\text{Var}}

\def\BF{\text{BF}}
\def\tr{\text{tr}}
\def\DKL{D_\text{KL}}
\def\SEN{\text{\it SEN}}
\def\SER{\text{SER}}
\def\FSER{F^{\text{SER}}}
\def\post{p_\text{post}}

\def\emul{\bar{\vmu}_l}
\def\emulsq{\widebar{\vmu_l^2}}
\def\emu{\bar{\vmu}}
\def\emusq{\widebar{\vmu^2}}
\def\ERSS{\text{ERSS}}

\def\er{\bar{\vr}_l}
\def\ebl{\E_{q_l}[\vb_l]}
\def\eb{\sum_l \ebl}

\def\ie{i.e.\,}
\def\eg{e.g.\,}


\begin{document}
\singlespacing
\title{\texttt{susieR} Implementation Details}
\author{Gao Wang, Yuxin Zou, Kaiqian Zhang}
\maketitle

\begin{abstract}
This document fills some technical gaps in \susie manuscript, mostly algebraic results for \susie VEM updates and ELBO calculation, and implementation of \susie in \texttt{susieR} with summary statistics and trend filtering interface and optimization.
\end{abstract}

\section{Notations}

We now describe the notation used in this text. We denote matrices by
boldface uppercase letters ($\mathbf{A}$), vectors are denoted by boldface
lowercase letters ($\mathbf{a}$), and scalars are denoted by non-boldface
letters ($a$ or $A$). All vectors are column-vectors. Lowercase
letters may represent elements of a vector or matrix if they have
subscripts. For example, $a_{ij}$ is the $(i,j)$th element of
$\mathbf{A}$, $a_i$ is the $i$th element of $\mathbf{a}$, and $\mathbf{a}_{i}$ is
either the $i$th row or $i$th column of $\mathbf{A}$. For indexing, we
will generally use capital non-boldface letters to denote the total
number of elements and their lowercase non-boldface versions to denote
the index. For example, $i = 1,\ldots,I$. We let $\mathbf{A}_{n \times p}$
denote that $\mathbf{A} \in \mathbb{R}^{n \times p}$. We denote the matrix
transpose by $\mathbf{A}^T$, the matrix inverse by $\mathbf{A}^{-1}$,
and the matrix determinant by $\det(\mathbf{A})$. Finally, sets will be
denoted by calligraphic letters ($\mathcal{A}$).

\section{\susie VEM updates}

This section derives equations (A.29) -- (A.36) in the manuscript. 

\subsection{Bayesian univariate regression} \label{sec:bur}

This corresponds to equations (2.4) -- (2.11) in the manuscript. Here I show results for (2.7) -- (2.11). By Bayes rule,
\begin{equation}
    \post(b|\vy, \rv, \rv_0) = \frac{p(\vy|\rv,b)p(b|\rv_0)}{p(\vy|\rv, \rv_0)}
\end{equation}
where the prior
\begin{equation}
    \log p(b|\rv_0) = -\frac{1}{2}\log (2\pi\rv_0) - \frac{1}{2\rv}b^2
\end{equation}
and log-likelihood
\begin{equation}
\log p(\vy|\rv, b) = -\frac{n}{2}\log(2\pi\rv) - \frac{1}{2\rv}\norm{\vy - \vx b}^2
\end{equation}

Given $\rv$ and $\rv_0$ the marginal log-likelihood is:
\begin{align}
    \log p(\vy|\rv, \rv_0) &= \log \int p(\vy|\rv, b) p(b|\rv_0)\dif b \label{eqn:bur_marginal}\\
    & = \int \big(-\frac{n}{2} \log(2\pi\rv) - \frac{1}{2\rv}\norm{\vy-\vx b}^2 - \frac{1}{2}\log(2\pi) - \frac{1}{2}\log\rv_0 - \frac{1}{2\rv_0}\norm{b}^2\big)\dif b \\
    & = -\frac{n}{2}\log (2\pi\rv) - \frac{1}{2}\log(2\pi) - \frac{1}{2}\log \rv_0 \\
    & -\frac{1}{2} \int \big( \frac{\vy^T\vy - 2b\vx^T\vy + \vx^T b^2\vx}{\rv} + \frac{b^2}{\rv_0}\big) \dif b \label{eqn:bur_kernel}
\end{align}

Let $\tau_n = \vx^T\vx + \frac{\rv}{\rv_0}$, $\mu_1 = \vx^T\vy / \tau_n = \vx^T\vx \hat{b} / \tau_n$ where $\hat{b}$ is OLS estimate of regression coefficient. Then
\begin{align}
    \eqref{eqn:bur_kernel} &= -\frac{1}{2}\int (\tau_nb^2 - 2\tau_n \mu_1 b + \vy^T\vy) / \rv \dif b \\ 
    &= -\frac{1}{2}\int \big[ \tau_n(b^2 - 2\mu_1 b + \mu_1^2) / \rv + (\vy^T\vy - \tau_n\mu_1^2) / \rv \big] \dif b \\ 
    &= \frac{1}{2} \int \big[\norm{b-\mu_1}^2 / (\rv/\tau_n) + (\vy^T\vy - \tau_n\mu_1^2) / \rv \big] \dif b
\end{align}

Let $\rv_1 = \rv / \tau_n$, 
\begin{align}
    \eqref{eqn:bur_marginal} &= -\frac{n}{2}\log(2\pi\rv) - \frac{1}{2} \log \rv_0 \\
    &+ \int \big[-\frac{1}{2} \log(2\pi\rv_1) - \frac{1}{2\rv_1}\norm{b-\mu_1}^2 \big] \dif b \\
    &+ \frac{1}{2}\log \rv_1 - \frac{\vy^T\vy}{2\rv} + \frac{\mu_1^2}{2\rv_1} \\
    &= -\frac{n}{2}\log(2\pi\rv) - \frac{1}{2} \log \rv_0 + \frac{1}{2}\log \rv_1 - \frac{\vy^T\vy}{2\rv} + \frac{\mu_1^2}{2\rv_1} \label{eqn:bur_marginal_result}
\end{align}

Posterior of $b$ is given by
\begin{equation}
b | \vy, \rv, \rv_0 \sim N(\mu_1, \rv_1)
\end{equation}

The log Bayes factor is
\begin{align}
    \log \BF := \log p(\vy|\rv, \rv_0) - \log p(\vy | \rv) \\
    &= \eqref{eqn:bur_marginal_result} - (-\frac{n}{2}\log(2\pi\rv) - \frac{\vy^T\vy}{2\rv}) \\
    &= - \frac{1}{2} \log \rv_0 + \frac{1}{2}\log \rv_1 + \frac{\mu_1^2}{2\rv_1}
\end{align}
Thus 
\begin{equation}
\BF(\vy, \vx; \rv, \rv_0) = (\rv_0/\rv_1)^{-\frac{1}{2}}\exp(\frac{\mu_1^2}{2\rv_1})
\end{equation}

\subsection{Bayesian single effect multiple regression} \label{sec:bser}

This section deals with model (2.12) -- (2.21) in the manuscript. Here we derive equation (2.18). For each single effect $j$ the posterior inclusion probability is
\begin{align}
    \valpha_j &= p(\gamma_j = 1 | \vy, \rv, \rv_0) \\
    &= \frac{p(\vy | \gamma_j = 1, \rv, \rv_0)p(\gamma_j=1)}{\sum_j p(\vy | \gamma_j = 1, \rv, \rv_0)p(\gamma_j=1)} \\
    &= \pi_j \frac{p(\vy | \gamma_j = 1, \rv, \rv_0) / p(\vy|\rv)}{\sum_j \pi_j p(\vy | \gamma_j = 1, \rv, \rv_0) / p(\vy | \rv)} \\
    &= \pi_j \frac{\BF(\vy, \vx_j; \rv, \rv_0)}{\sum_j \pi_j \BF(\vy, \vx_j; \rv, \rv_0)}
\end{align}
Derivations of single effect Bayes factors and posteriors follow directly from Section \ref{sec:bur}. We have now completed derivation of equations (A.29) -- (A.36) of the manuscript, the core updates of the \susie VEM algorithm.

\section{ELBO updates}

This section shows derivation of equation (A.19) for \susie model. It involves the ERSS in (A.19), the conditional likelihood in (A.40) and the marginal likelihood in (A.46) for Bayesian single effect model in Section \ref{sec:bser}. 

\subsection{ERSS}

Analytic form of ERSS can be obtained by putting together (A.22), (A.24), (A.26), (A.37) and (A.38). (A.22) is derived in (A.48).

\subsection{Marginal likelihood for the $l$-th effect}

For single effect model,
\begin{align}
    \log L(\vy; \rv, \rv_0) &= \log \int p(\vy | \vb, \rv, \rv_0) p(\vb | \rv, \rv_0) \dif \vb\\
    &= \log \iint p(\vy | \vb, \vgamma, \rv, \rv_0) p(\vb | \vgamma, \rv_0) p(\vgamma | \vpi)  \dif \vb \dif \vgamma \\
    &= \log \int \sum_j p(\vy | b_j, \rv, \rv_0) p(b_j|\gamma_j = 1, \rv_0) p(\gamma_j = 1 | \vpi) \dif b_j \\
    &= \log \sum_j \pi_j \int N(\vy; \vx_jb_j, \rv I_n) N(b_j; 0, \rv_0) \dif b_j \\
    &= \log \sum_j \pi_j p(\vy|\vx_j, \rv, \rv_0) \\
    &= \log \sum_j \pi_j \big[\BF(\vy, \vx_j; \rv, \rv_0)p(\vy|\rv) \big]\\
    &= \log N(y; 0, \rv I_n) + \log \sum_j \pi_j \BF(\vy, \vx_j; \rv, \rv_0)
\end{align}

\subsection{Conditional likelihood for the $l$-th effect}

To compute conditional likelihood for single effect model we are left to deal with $\E_{q_l}\norm{\vy - \vmu_l}^2$ in Equation (A.46) in the manuscript,
\begin{align}
    \E_{q_l}\norm{\vy - \vmu_l}^2 &= \E_{q_l}\norm{\vy - \mx\vb_l}^2 \\
    &= \vy^T\vy - 2\vy^T\mx \E_{q_l}[\vb_l] + \mx^T\mx \E_{q_l}[\vb_l^T\vb_l]
\end{align}
where the expectations (first and second moments) are given by (A.37) and (A.38) in the manuscript.

\section{Update residual variance}
This is a crucial step but is trivial to derive for univariate case ($\vy$ is a vector). It is given by Equation (A.23) in the manuscript.

\section{\susie updates using summary statistics}
\section{\susie for trend filtering}
\subsection{Overview}

Trend filtering is a useful statistical tool for nonparametric regression. \cite{Kim07l1trend} first proposed $\ell_1$ trend filtering for estimating underlying piecewise linear trends in time series data. This idea can be further extended to fit piecewise polynomial of degree $k$ to the data. In their paper, Kim et al. showed the equivalence between the $\ell_1$ trend filtering and the $\ell_1$-regularized least squares problem. This motivates us to think about the connection between trend filtering and sparse approximation in general. 

\subsection{Trend filtering and sparse regression}
Trend filtering problem is defined mathematically as follows. For a given integer $k \geq 0$, the kth order trend filtering is defined by a penalized least squares optimization problem,
\begin{align}
\hat{\vb} = \underset{\vb}{\mathrm{argmin}} \frac{1}{2}|| \vy- \vb||_2^2 + \frac{n^k}{k!}\lambda||D^{(k+1)}\vb||_1,
\end{align}
where $\vy = [y_1 \dots y_n]^T$ is an n vector of observations, $\lambda$ is a tuning parameter, and $D^{(k+1)}$ is the discrete difference operator of order $k$. When order $k=0$, $D$ is defined 

\begin{align}
D^{(1)} = \begin{bmatrix} 
    -1 & 1 & 0 & \dots & 0 & 0\\
    0 & -1 & 1 & \dots & 0 & 0\\
    \vdots & \ddots & \\
    0 & 0 & 0 & \dots & -1 & 1\\
    \end{bmatrix}
    \in \mathbb{R}^{(n-1)\times n}.
\end{align}

In this case, the components of the trend filtering estimate form a piecewise constant structure, with break points corresponding to the nonzero entries of $D^{(1)}\hat{\vb} = (\hat{b}_2 - \hat{b}_1, \dots, \hat{b}_n - \hat{b}_{n-1})$ \cite{Tibshirani2014}. And when $k\geq 1$, the operator $D^{(k+1)}$ is defined recursively,

\begin{align}
D^{(k+1)} = D^{(1)} \cdot D^{(k)} \in \mathbb{R}^{(n-k-1)\times n}.
\end{align}
Now we want to transform the trend filtering problem into a sparse regression problem. Let $\bm{\beta}=D^{(k+1)}\vb$. Then if $D^{(k+1)}$ were invertibe, we could write $\vb = (D^{(k+1)})^{-1}\bm{\beta}$ and the above problem would become

\begin{align}
\hat{\bm{\beta}} = \underset{\bm{\beta}}{\mathrm{argmin}} \frac{1}{2}||\vy- (D^{(k+1)})^{-1}\bm{\beta}||_2^2 + \frac{n^k}{k!}\lambda||\bm{\beta}||_1.
\end{align}
We can consider this a sparse regression with $\ell_1$ regularization problem, where design matrix $X$ is $(D^{(k+1)})^{-1}$. 

\subsection{Modification on $D$}
As we have seen, the trend filtering problem becomes a sparse regression with $\ell_1$ regularization if we consider the design matrix  $X = (D^{(k+1)})^{-1}$. However, $D^{(1)} \in \mathbb{R}^{(n-1)\times n}$ is not invertible, so is $D^{(k+1)}$ for $k=1,2,\dots$. By observation, we complete $D^{(1)}$ as a square and symmetric matrix  

\begin{align}
\hat{D}^{(1)} = \begin{bmatrix} 
    -1 & 1 & 0 & \dots & 0 & 0\\
    0 & -1 & 1 & \dots & 0 & 0\\
    \vdots & \ddots & \\
    0 & 0 & 0 & \dots & -1 & 1\\
    0 & 0 & 0 & \dots & 0 & -1
    \end{bmatrix}
    \in \mathbb{R}^{n\times n}.
\end{align}
And for $k\geq 1$, we obtain
\begin{align}
\hat{D}^{(k+1)} = \hat{D}^{(1)} \cdot \hat{D}^{(k)} \in \mathbb{R}^{n\times n}.
\end{align}
We notice that, by this modification, $\hat{D}^{(k+1)}$ has $k$ more rows added at the bottom without changing any previous entry. With this modification, we are able to invert $\hat{D}^{(k+1)}$ and consider the inverse matrix as an $X$ matrix in the sparse regression. 

\subsection{Special structure on $\hat{D}^{-1}$}
After determining the design matrix $X$ in the sparse regression problem, we could apply \susie algorithm to help us find a possible fit. Rather than generating $\hat{D}^{-1}$ and set this as an $X$ input, we exploit the special structure of $\hat{D}^{-1}$ and perform \susie on this specific trend filtering problem with $O(n)$ complexity. We will talk about how to make different computations linear in complexity by utilizing the special structure respectively. 

\subsection{Computation on $Xb$} \label{Computation on Xb}
In the trend filtering application, since $X = (\hat{D}^{(k+1)})^{-1}$, we obtain
\begin{align}
X \vb & = (\hat{D}^{(k+1)})^{-1} \vb = (\underbrace{\hat{D}^{(1)}\dots \hat{D}^{(1)}}_{k+1})^{-1} \vb \\
 & = \underbrace{(\hat{D}^{(1)})^{-1} \dots (\hat{D}^{(1)})^{-1}}_{k+1} \vb. 
\end{align}
We notice that since 
\begin{align}
(\hat{D}^{(1)})^{-1} = \begin{bmatrix} 
    -1 & -1 & -1 & \dots & -1 & -1\\
    0 & -1 & -1 & \dots & -1 & -1\\
    \vdots & \ddots & \\
    0 & 0 & 0 & \dots & -1 & -1\\
    0 & 0 & 0 & \dots & 0 & -1
    \end{bmatrix}
    \in \mathbb{R}^{n\times n},
\end{align}
 
\begin{align}
(\hat{D}^{(1)})^{-1} \vb & = -1 \cdot [b_1+b_2+\dots+b_n, b_2+\dots+b_n, \dots, b_{n-1}+b_n, b_n]^T \\
& = -1 \cdot \text{cumsum}(\text{reverse}(\vb)) .
\end{align}
Let $f: \mathbb{R}^n \to \mathbb{R}^n$ such that $f(\vx)= -\text{cumsum}(\text{reverse}(\vx))$ for any $\vx \in \mathbb{R}^n$. Then
\begin{align}
X \vb & = (\hat{D}^{(k+1)})^{-1} \vb = f^{(k+1)}(\vb),
\end{align}
where $k$ is the order of trend filtering. 

\subsection{Computation on $X^T y$} \label{Computation on Xty}
We consider $X^T\vy$. Here $X = (\hat{D}^{(k+1)})^{-1}$ in the trend filtering problem, and $\vy$ is an n vector. We have

\begin{align}
X^T \vy & = ((\hat{D}^{(k+1)})^{-1})^T \vy = ((\underbrace{\hat{D}^{(1)}\dots \hat{D}^{(1)}}_{k+1})^{-1})^T \vy \\
& = \underbrace{((\hat{D}^{(1)})^{-1})^T \dots ((\hat{D}^{(1)})^{-1})^T}_{k+1} \vy.
\end{align}
Similarly, we observe that since
\begin{align}
((\hat{D}^{(1)})^{-1})^T = \begin{bmatrix} 
    -1 & 0 & 0 & \dots & 0 & 0\\
    -1 & -1 & 0 & \dots & 0 & 0\\
    \vdots & \ddots & \\
    -1 & -1 & -1 & \dots & -1 & 0\\
    -1 & -1 & -1 & \dots & -1 & -1
    \end{bmatrix}
    \in \mathbb{R}^{n\times n},
\end{align}

\begin{align}
X^T \vy & = -1 \cdot [y_1, y_1+y_2, \dots, y_1+y_2+\dots+y_n]^T \\
& = -1 \cdot \text{cumsum}(\vy). 
\end{align}
Let $g: \mathbb{R}^n \to \mathbb{R}^n$ such that $g(\vx)= -\text{cumsum}(\vx)$ for any $\vx \in \mathbb{R}^n$. Then
\begin{align}
X^T \vy & = ((\hat{D}^{(k+1)})^{-1})^T \vy = g^{(k+1)}(\vy),
\end{align}
where $k$ is the order of trend filtering.

\subsection{Computation on $(X^2)^T 1$ (i.e. colSums($X^2$))} \label{Computation on d}

To compute $(X^2)^T \bm{1}$, where $X = (\hat{D}^{(k+1)})^{-1}$, let's first explore the special structure of $(\hat{D}^{(k+1)})^{-1}$ for $k=0,1,2$. 
\begin{align}
(\hat{D}^{(1)})^{-1} = \begin{bmatrix} 
    -1 & -1 & -1 & -1 & -1 & \dots \\
    0  & -1 & -1 & -1 & -1 & \dots \\
    0  & 0  & -1 & -1 & -1 & \dots \\
    0  & 0  & 0  & -1 & -1 & \dots \\
    \vdots & \ddots & \\
    \end{bmatrix}
    \in \mathbb{R}^{n\times n},
\end{align}

\begin{align}
(\hat{D}^{(2)})^{-1} = \begin{bmatrix} 
    1  & 2 & 3 & 4 & 5 & 6 & \dots \\
    0  & 1 & 2 & 3 & 4 & 5 & \dots \\
    0  & 0 & 1 & 2 & 3 & 4 &\dots \\
    0  & 0 & 0 & 1 & 2 & 3 &\dots \\
    \vdots & \ddots & \\
    \end{bmatrix}
    \in \mathbb{R}^{n\times n},
\end{align}

\begin{align}
(\hat{D}^{(3)})^{-1} = \begin{bmatrix} 
    -1 & -3 & -6 & -10 & -15 & \dots \\
    0  & -1 & -3 & -6  & -10 & \dots \\
    0  &  0 & -1 & -3  & -6  & \dots \\
    0  &  0 & 0  & -1  & -3  & \dots \\
    \vdots & \ddots & \\
    \end{bmatrix}
    \in \mathbb{R}^{n\times n},
\end{align}

Define a triangular rotate matrix $X \in \mathbb{R}^{n\times n}$ such that 

(i) For any $i, j \leq n$, $X_{ij} = 0$ if $i>j$.

(ii) For any $k < n$, $X_{ab} = X_{cd}$ if $b-a = d-c = k$. 


We observe that if $X$ is a triangular rotate matrix, then 

\begin{align}
X^T \bm{1} = \text{cumsum}(X_{1.}).
\end{align}

Since $X^2$ is still a triangular rotate matrix, we obtain

\begin{align}
(X^2)^T \bm{1} = \text{cumsum}(X^2_{1.}).
\end{align}

Since $X = (\hat{D}^{(k+1)})^{-1}$ is a triangular rotate matrix, 

\begin{align}
(X^2)^T \bm{1} = \text{cumsum}(((\hat{D}^{(k+1)})^{-1})_{1.}^2).
\end{align}

And obviously, the first row of $(\hat{D}^{(k+1)})^{-1}$ is 
\begin{equation}
 ((\hat{D}^{(k+1)})^{-1})_{1.} = 
    \begin{cases} 
      \bm{-1} & \text{if } k = 0 \\
      g^{(k)}(\bm{1}) & \text{if } k > 0.
   \end{cases}
\end{equation}

\subsection{Computation on $\frac{1}{n}X^T 1$ (i.e. column means)} \label{Computation on cm}

By \ref{Computation on d}, we know that given that $X = (\hat{D}^{(k+1)})^{-1}$,
\begin{align}
 \frac{1}{n} X^T \bm{1} = \frac{1}{n} \text{cumsum}(X_{1.}) = \frac{1}{n} \text{cumsum}(((\hat{D}^{(k+1)})^{-1})_{1.}),
\end{align}
where $((\hat{D}^{(k+1)})^{-1})_{1.}$ is defined above in \ref{Computation on d}. 

\subsection{Computation on $s$ (i.e. column standard deviations)}
We consider for each column $j$, $j=1,2,\dots, n$,

\begin{align}
s_j & = \sqrt{E[X_{.j}^2] - E[X_{.j}]^2} \\
       & = \sqrt{\frac{1}{n}\sum_{i=1}^{n}X_{ij}^2 - (\frac{1}{n}\sum_{i=1}^{n} X_{ij})^2}.
\end{align}
Hence,

\begin{align}
\bm{s} & = \sqrt{E[X^2] - E[X]^2} \\
       & = \sqrt{\frac{1}{n}\text{colSums}(X^2) - (\frac{1}{n}\text{colSums}(X))^2} \\
       & = \sqrt{\frac{1}{n}(X^2)^T \bm{1} + (\frac{1}{n} X^T \bm{1})^2},
\end{align}
where the first term involves \ref{Computation on d} and the second term is computed in \ref{Computation on cm}.

\subsection{Computation on $(\hat{X}^2)^T 1$, where $\hat{X}$ is standardized} \label{Computation on std_d}

Given that $X = (\hat{D}^{(k+1)})^{-1}$. We first standardize $X$ to get $\hat{X}$. Our goal is to compute $(\hat{X}^2)^T \bm{1}$. By observation, 
\begin{equation}
(\hat{X}^2)^T \bm{1} = 
\begin{cases} 
      [\underbrace{n-1, n-1, \dots, n-1}_{n-1}, 0]^T & \text{if } k = 0 \\
      [\underbrace{n-1, n-1, \dots, n-1, n-1}_{n}]^T & \text{if } k \neq 0. 
\end{cases}
\end{equation}

\subsection{Conclusion}

Computation details from section \ref{Computation on Xb} to section \ref{Computation on std_d} explain how we can benefit from the unique structure of matrices from trend filtering problem. As shown by our formula, we do not need to form any matrix and complete \susie algorithm with $O(n)$ complexity. 


















\end{document}
